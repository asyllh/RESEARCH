\documentclass[oribibl]{llncs}

%---- PREAMBLE ----%

% Allows separate files to be used in the main file
\usepackage{subfiles}

% Algorithms
\usepackage{algorithm}
\usepackage[noend]{algpseudocode}
\renewcommand{\algorithmicrequire}{\textbf{Input:}}
\renewcommand{\algorithmicensure}{\textbf{Output:}}
\newcommand{\IndStatex}[1][1]{\Statex\hspace{6mm}}
\newcommand{\IndIndStatex}[1][1]{\Statex\hspace{12mm}}
\newcommand{\IndIndIndStatex}[1][1]{\Statex\hspace{18mm}}

% Contains advanced math extensions
\usepackage{amsmath}

% Introduces the *proof* environment and the \theoremstyle command
\usepackage{amsthm}

% Adds new symbols to be used in math mode, e.g. \mathbb
\usepackage{amssymb}

% To declare multiple authors
\usepackage{authblk}

% Provides extra comands as well as optimisation for producing tables
\usepackage{booktabs}

% Allows customisation of appearance and placements for figures/tables etc.
\usepackage{caption}

% Adds support for arbitrarily-deep nested lists
\usepackage{enumitem}

% Improves the interface for defining floating objects such as figures/tables
\usepackage{float}

\usepackage{fullpage}

% For easy management of document margins and the document page size
%\usepackage{geometry}

% Allows insertion of graphic files within a document
\usepackage{graphicx}

% Manage links within the document or to any URL when you compile in PDF
\usepackage{hyperref} 

% Successor of amsmath
\usepackage{mathtools}

% No indentation, space between paragraphs
\usepackage{parskip}

%Include standalone .tex files
\usepackage{standalone}

% Define multiple floats (figures/tables) within one environment with individual captions 1a, 1b etc
\usepackage{subcaption}

\usepackage{tikz}

\usepackage{wrapfig}

%Theorem style
\theoremstyle{plain}% default
\newtheorem{theorem}{Theorem}[section]
\newtheorem{corollary}{Corollary}[theorem]

\theoremstyle{definition}
\newtheorem{defn}{Definition}[section]
\newtheorem{proposition}{Proposition}[defn]
\newtheorem{exmp}{Example}[section]

\title{Heuristics for the Score-Constrained Bin-Packing Problem}
\author{Asyl L. Hawa \and Rhyd M. R. Lewis \and Jonathan M. Thompson}
\institute{School of Mathematics, Cardiff University, Senghennydd Road, Cardiff, UK, CF24 4AG}
\date{\today}

\begin{document}
\maketitle

\begin{abstract}
	
\end{abstract}

\section{Introduction}
\label{sec:intro}

\begin{definition}
	\label{defn:cpop}
	Let $\mathcal{M}$ be a multiset of unordered pairs of positive integers $\mathcal{M} = \{\{a_1, b_1\}, \{a_2,b_2\},...,\{a_n,b_n\}\}$, and let $\mathcal{T}$ be an ordering of the elements of $\mathcal{M}$ such that each element is a tuple. The Constrained Pair Ordering Problem (CPOP) consists of finding a solution $\mathcal{T}$ such that, given a fixed value $\tau \in \mathbb{R}^{+},$
	\begin{equation}
		\label{eqn:cpop}
		\textup{\textbf{rhs}}(i) + \textup{\textbf{lhs}}(i+1) \geq \tau \hspace{5mm} \forall \hspace{1mm} i \in \{1,2,..., n-1\},
	\end{equation}
	where \textup{\textbf{lhs}($i$)} and \textup{\textbf{rhs}($i$)} denote the left- and right-hand values of the $i$th tuple.	
\end{definition}

For example, given the multiset $\mathcal{M} = \{\{1,2\}, \{1,5\}, \{2,4\}, \{3,4\}, \{4,5\}\}$ and $\tau = 7$, one possible feasible solution is $\mathcal{T} = \langle(1,2), (5,4), (3,4), (4,2), (5,1) \rangle$.

\textcolor{OrangeRed}{Model the problem using a graph.}

Let $\mathcal{M}$ be a multiset as described in Definition \ref{defn:cpop}, with an additional unordered pair consisting of ``dominating'' values equal to $\tau$. Two values that make up a pair in $\mathcal{M}$ are referred to as ``partners''. Then, $G(V, P \cup E)$ is an undirected graph defined by a vertex set $V = \{v_1, v_2, ...v_{2n+2}\}$, where each vertex is assigned a value from $\mathcal{M}$ such that $v_1 \leq v_2 \leq ... \leq v_{2n+2}$. $G$ comprises two distinct edge sets: $P$, which contains edges between vertices that are partners, and $E$, which consists of edges between vertices that fulfil the \textit{joint sum constraint}. That is, for two distinct vertices $v_i, v_j$,

\begin{equation}
	\label{eqn:jointsum}
	v_i + v_j \geq \tau,
\end{equation}

provided $\{v_i, v_j\} \notin P$. Observe that $P$ is in fact a perfect matching, with $|P| = |\mathcal{M}| = n+1$, and that $P \cap E = \emptyset$.

If $G(V, P\cup E)$ features an alternating Hamiltonian cycle, then a feasible solution $\mathcal{T}$ exists.

\begin{definition}
	\label{defn:althamcycle}
	Let $G(V, P\cup E)$ be an undirected graph, where each edge is a member of one of two sets, $P$ or $E$. $G$ contains an alternating Hamiltonian cycle if there exists a Hamiltonian cycle such successive edges alternate between sets $P$ and $E$. For example, if $\{v_{i-1}, v_i\} \in P$, then $\{v_i, v_{i+1}\} \in E$, or vice versa.
\end{definition}

\begin{theorem}
	\label{thm:cpopsolnalt}
	There exists a feasible solution $\mathcal{T}$ to an instance $\mathcal{M}$ of the CPOP if and only if its corresponding graph $G(V, P\cup E)$ contains an alternating Hamiltonian cycle.
\end{theorem}
\begin{proof}
	To be completed. \qed
\end{proof}

The values in each unordered pair in $\mathcal{M}$ cannot be altered, and so every edge in $P$ must be present in the alternating Hamiltonian cycle. Thus, the task involves finding a matching $Q \subset E$ of cardinality $n+1$, such that the edge sets $P$ and $Q$ form an alternating Hamiltonian cycle as described in Definition \ref{defn:althamcycle}. If no suitable matching $Q$ can be found, then, in accordance with Theorem \ref{thm:cpopsolnalt}, no feasible solution exists for the given instance $\mathcal{M}$.

To find a matching $Q \subset E$, the following modified maximum cardinality matching algorithm is executed (\citealp{mahadev1994, becker2010}): for each vertex $v_i, i = 1, 2,...,2n+2$, find the largest vertex in $v_i$'s neighbourhood, excluding partners, that is not incident to any edge in $Q$. Add the edge connecting these two vertices to $Q$, and proceed to the next vertex until all vertices have been assessed. The pair of vertices incident to each edge in $Q$ are referred to as being ``matched'', and are connected by a ``matching'' edge. 

During this algorithm, if a vertex $v_i$ is not adjacent to any other unmatched vertex except its partner, the preceeding vertex $v_{i-1}$ can be rematched, provided that $v_i$ is not the first vertex (i.e $i \neq 1$), $v_{i-1}$ is incident to an edge in $Q$, (i.e. $v_{i-1}$ has been matched) and $v_{i-1}$ is adjacne to $v_i$'s partner via an edge in $E$. Thenm, $v_{i-1}$ is matched with $v_i$'s partner, and $v_i$ is matched with $v_{i-1}$'s previous match. It is guaranteed that this algorithm will produce a maximum cardinality matching $Q \subset E$ (\citealp{becker2010}).

If $|Q| < n+1$, clearly there are not enough matching edges to form an alternating Hamiltonian cycle along with the edges in $P$, and hence there is no feasible solution to this particular instance of the CPOP. However, if $|Q| = n+1$, then the spanning subgraph $G'(V, P \cup Q)$ is a 2-regular graph, where each vertex is adjacent to its partner via an edge in $P$, and its match via an edge in $Q$. $G'$ thus consists of a collection of alternating Hamiltonian cycles $A_1, A_2, ..., A_{\lambda}$. If there is only one alternating Hamiltonian cycle on $G'$, i.e. $\lambda = 1$, then a feasible solution has been found \textcolor{OrangeRed}{remove dominating vertices}. If $\lambda > 1$, then the components of $G'$ need to be combined into one single alternating Hamiltonian cycle, by removing some matching edges from $Q$, and adding new edges from $E$. The following algorithm finds edges that are able to merge the components of $G'$ together.

For each edge $(v_i, v_j) \in Q$, with $i < j$, $v_i$ is referred to as the ``smaller'' vertex, and $v_j$ as the ``larger'' vertex, since the value associated with $v_i$ is less than or equal to the value associated with $v_j$. Initially, the edges in $Q$ are sorted in a list such that the smaller vertices of the edges are in lexicographical order, and therefore in non-decreasing order of values. \textcolor{OrangeRed}{remove edges from list that are a result of rematching}. Starting from the first edge, search through the list to find an edge that meets the following conditions:
\begin{enumerate}
	\item The smaller vertex of the current edge is adjacent to the larger vertex of the succeeding edge in the list via an edge in $E$,
	\item The current edge and the succeeding edge are members of different alternating Hamiltonian cycles on $G$.
\end{enumerate}

If such an edge is found, add it to a set $C_i$, and continue to add all succeeding edges in the list to $C_i$, provided that for each edge conditions 1 and 2 hold, and the succeeding edge does not belong to a component on $G$ that is already represented by an edge in $C_i$. Once there are no more valid edges to add to $C_i$, continue to search through the remaining edges in the list to find an edge that meets conditions 1 and 2, and can therefore start a new set $C_{i+1}$. If the final edge in the list has been reached, the process is terminated.

The sets $C_i, i = 1, ..., k$ that have been created during this process are then analysed. In the case where not a single set $C$ has been produced, then the cycles $A_1,..., A_{\lambda}$ of $G'$ cannot be combined, and therefore no solution exists. If there exists a set $C_i$ such that $|C_i| = \lambda$, then all components on $G'$ can be combined as follows:

Starting from the first edge in $C_i$, add the edge from $E$ connecting the smaller vertex of each edge to the larger vertex of the succeeding edge to the matching $Q$. For the final edge in $C_i$, add the edge from $E$ that connects its smaller vertex to the larger vertex of the first edge in $C_i$ to $Q$. Then, edges that appear in both $C_i$ and $Q$ are removed from $Q$, such that $C_i \cap Q = \emptyset$. \textcolor{OrangeRed}{Then components have been merged, remove dominating vertices etc}.


If there is no set $C_i$ such that $|C_i| = \lambda$, and $k > 1$ (i.e. there are multiple sets $C_i, i = 1,..., k$), it may be that multiple sets are required to combine the components of $G'$. Let $\mathcal{C}$ be a set that will contain all the sets $C_i$ needed to merge all components of $G'$ together, and let $\Lambda$ be a set that records the components covered by the sets in $\mathcal{C}$. For the set $C_1$, append the componnents $A_i$ that each edge in $C_1$ is a member of to $\Lambda$, and then add $C_1$ to $\mathcal{C}$. At this stage, $|\Lambda| = |\mathcal{C}|$. Proceeding throug the sets $C_i, i = 2, ..., k$, find a set $C_i$ that contains at least one edge that is a member of a component recorded in $\Lambda$, and at least one edge that is a member of a component not yet recorded in $\Lambda$. Append the component that each edge in this set $C_i$ is a member of to $\Lambda$, provided the component is not already a member of $\Lambda$, and add $C_i$ to $\mathcal{C}$. Continue in this manner until the final set $C_k$ has been assessed. If $|\Lambda| = \lambda$, then the sets in $\mathcal{C}$ are able to merge the components of $G'$ together. This is done by repeating the connecting process for a single set $C_i$ on every set in $\mathcal{C}$. 

\textcolor{OrangeRed}{Complexity of the CPOP algorithm, ability to find a solution, complete recognition.}

One prominent application of the CPOP is in a cutting stock problem \textcolor{OrangeRed}{(or strip packing problem)} brought to light as an open-combinatorial problem by \cite{goulimis2004} in 2004. A set $\mathcal{I}$ of rectangular items of equal height $h$ made from cardboard are to be packed onto  strip from left to right. Each item $i \in \mathcal{I}$ has a width $w_i \in \mathbb{Z}^{+}$, and possesses two vertical score lines, marked in predetermined placed. A pair of knives mounted on a bar cuts along the score lines of two adjacent items simultaneously, which allows the items to be folded with ease (see Figure ref{}). However, by desgn, the scoring knives cannot be placed too close to one another, and as such have a ``minimum scoring distance'' (around 70mm in industry). Thus, the distance between two score lines on adjacent items must be greater than or equal to the minimum scoring distance in order for the knives to be able to cut the items correctly. The distance between a score line and the nearest edge on an item is called a score width. Each item $i \in \mathcal{I}$ thus has two score widths, $a_i, b_i \in \mathbb{Z}^{+}$, with $a_i \leq b_i$. Since these score widths are not necessarily equal, an item $i$ can be positioned in one of two orientations: ``regular'', denoted $(a_i, b_i)$, where the smaller score $a_i$ is on the left-hand side of the item, or ``rotated'', denoted $(b_i, a_i)$, where the item has been rotated by $180^{\circ}$ so that the smaller score $a_i$ is on the right-hand side of the item. Then, for two items to be feasibly placed alongside one another, the sum of the adjacent score widths must be greater than or equal to the minimum scoring distance. Thus, the problem consists of finding a suitable ordering and orientation of the items on a strip $\mathcal{S}$ that fulfils the \textit{minimum total score width constraint},

\begin{equation}
\label{eqn:mtswc}
\textup{\textbf{rhs}}(i) + \textup{\textbf{lhs}}(i+1) \geq \tau \hspace{5mm} \forall \hspace{1mm} i \in \{1,2,..., |\mathcal{S}|-1\},
\end{equation}

where \textup{\textbf{lhs}($i$)} and \textup{\textbf{rhs}($i$)} denote the left- and right-hand values of the $i$th item on the strip $\mathcal{S}$, and $\tau \in \mathbb{R}^{+}$ is the minimum scoring distance.

It can be seen that this problem is analogous to the CPOP, where the pairs of score widths on each item are expressed as an unordered pair in a multiset $\mathcal{M}$, and therefore the problem can be solved in the same manner.

In the problem detailed by Goulimis, the aim is to align all of the items onto a single strip. So, when an instance of this problem is expressed as an instance of the CPOP, the widths of the individual items are not taken into account. In reality, strips of material are provided in fixed lengths. Given a large problem instance, multiple strips may be required to feasibly accommodate all of the items. For this reasons, we define a new problem to be investigated.

\begin{definition}
	Given a set $\mathcal{I}$ of rectangular items of varying widths $w_i$ and score widths $a_i, b_i$ for each item $i \in \mathcal{I}$, a finite set of strips $S_j$ with equal fixed length $l$, and a minimum scoring distance $\tau$, the Score-Constrained Strip-Packing Problem (SCSPP) consists of finding the minimum number of strips required to pack all items in $\mathcal{I}$ such that the minimum total score width constraint is met between every pair of adjacent items on every strip.
\end{definition}

Indeed, the SCSPP is a generalisation of the classical one-dimensional bin-packing problem (BPP), where in the latter problem, the minimum scoring distance $\tau$ can be said to be equal to zero. It follows that the SCSPP is at least as hard as the BPP, which is known to be NP-hard (\citealp{garey1979}), and so (under the assumption that $P \neq NP$) there is no known algorithm that is able to find an optimal solution for every instance of the SCSPP in polynomial time. Instead, heuristics can be used to find near-optimal solutions in a shorter amount of time. One example is the greedy heuristic known as first-fit (FF), an online algorithm that places each item (in some arbitrary order) onto the lowest-indexed strip such that the capacity of the strip is not exceeded. It has been shown that by arranging the items in decreasing order of size, a packing heuristic will produce a solution that is closer to the optimal than if the items are arranged in any other order (\citealp{johnson1974fast}). Applying FF onto items sorted in this manner yields the well-known first-fit decreasing (FFD) heuristic.

An optimal solution of an instance of the SCSPP is a solution that consists of the fewest number of strips, $k$. A simple lower bound for $k$ that can be computed in $O(n)$ time is given by

\begin{equation}
	k \geq \ceil*{\frac{\sum_{i=1}^{n}w_i}{l}}
\end{equation}

where $n$ is the number of items in the set $\mathcal{I}$, $w_i$ the width of each item $i \in \mathcal{I}$, and $l$ is the length of the strips (\citealp{martello1990b}). In 1973, \citeauthor{johnson1973} showed that FFD is guaranteed to return a solution that uses no more than $\frac{11}{9}k + 4$ strips. More recently, \cite{dosa2007} has proven that the worst case for FFD is in fact $\frac{11}{9}k + \frac{6}{9}$, and that this bound is tight. Due to the initial sorting of the $n$ items in non-increasing order of sizes, the time complexity of FFD is $O(n\log n)$.


















































%========================================================================%


\begin{comment}
The PSP can be seen in a problem related to the cutting-stock problem, first described by \cite{goulimis2004}. Let $\mathcal{I}$ be a set of rectangular items of equal heights, made from corrugated cardboard. Each item $i \in \mathcal{I}$ has width $w_i \in \mathbb{Z}^{+}$ and possess two vertical score lines. A pair of knives mounted on a bar will cut along these score lines to allow the items to be folded with ease. The distances between each edge of an item and the nearest score line are the score widths, $a_i, b_i \in \mathbb{Z}^{+}$, assigned such that $a_i \leq b_i$. As illustrated in Figure \ref{fig:boxesknife}, the knives cut along score lines from two adjacent items simultaneously. By design, the knives cannot be placed too close to one another - the distance between the knives must exceed a fixed minimum $\tau \in \mathbb{R}^{+}$ (around 70mm in industry). In order for the knives to be able to cut the items in the correct locations, two adajcent score widths, for example $a_i$ and $b_{i+1}$, must fulfil the \textit{minimum total score width constraint}

\begin{equation*}
a_i + b_{i+1} \geq \tau.
\end{equation*}

That is, the sum of two adjacent score widths must exceed the \textit{minimum total score width} $\tau$. Therefore, the problem involves finding a feasible arrangement of the items such that between each pair of items the minimum total score width constraint is met.


\begin{figure}[H]	
\centering
\includestandalone[width=0.4\textwidth]{figures/boxesknifeannotate}
\caption{\textcolor{red}{Individual boxes.}}	
\label{fig:boxesknife}
\end{figure}

Since the score widths on each item $i \in \mathcal{I}$ are not necessarily equal, they can be aligned in two distinct orientations: ``regular'', denoted by $(a_i, b_i)$, and ``rotated'', denoted by $(b_i, a_i)$, where the smaller of the two score widths $a_i$ is on the left- and right-hand side of the item respectively.

This can be seen as a PSP, where the values $x_i$ and $y_i$ are the score widths, and a feasible solution $\mathcal{X}$ consists of an arrangement of items in some order such that $\textup{\textbf{rhs}}(i) + \textup{\textbf{lhs}}(i+1) \geq \tau \forall  i \in \{1,2,..., |\mathcal{I}|-1\}$.

There exists a polynomial-time algorithm that is able to recognise whether a particular instance of this problem is feasible (\citealp{becker2010}).

\end{comment}



\bibliographystyle{dcu}
\bibliography{includes/bibliography}

\end{document}