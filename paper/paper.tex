\documentclass[oribibl]{llncs}

%---- PREAMBLE ----%

% Allows separate files to be used in the main file
\usepackage{subfiles}

% Algorithms
\usepackage{algorithm}
\usepackage[noend]{algpseudocode}
\renewcommand{\algorithmicrequire}{\textbf{Input:}}
\renewcommand{\algorithmicensure}{\textbf{Output:}}
\newcommand{\IndStatex}[1][1]{\Statex\hspace{6mm}}
\newcommand{\IndIndStatex}[1][1]{\Statex\hspace{12mm}}
\newcommand{\IndIndIndStatex}[1][1]{\Statex\hspace{18mm}}

% Contains advanced math extensions
\usepackage{amsmath}

% Introduces the *proof* environment and the \theoremstyle command
\usepackage{amsthm}

% Adds new symbols to be used in math mode, e.g. \mathbb
\usepackage{amssymb}

% To declare multiple authors
\usepackage{authblk}

% Provides extra comands as well as optimisation for producing tables
\usepackage{booktabs}

% Allows customisation of appearance and placements for figures/tables etc.
\usepackage{caption}

% Adds support for arbitrarily-deep nested lists
\usepackage{enumitem}

% Improves the interface for defining floating objects such as figures/tables
\usepackage{float}

\usepackage{fullpage}

% For easy management of document margins and the document page size
%\usepackage{geometry}

% Allows insertion of graphic files within a document
\usepackage{graphicx}

% Manage links within the document or to any URL when you compile in PDF
\usepackage{hyperref} 

% Successor of amsmath
\usepackage{mathtools}

% No indentation, space between paragraphs
\usepackage{parskip}

%Include standalone .tex files
\usepackage{standalone}

% Define multiple floats (figures/tables) within one environment with individual captions 1a, 1b etc
\usepackage{subcaption}

\usepackage{tikz}

\usepackage{wrapfig}

%Theorem style
\theoremstyle{plain}% default
\newtheorem{theorem}{Theorem}[section]
\newtheorem{corollary}{Corollary}[theorem]

\theoremstyle{definition}
\newtheorem{defn}{Definition}[section]
\newtheorem{proposition}{Proposition}[defn]
\newtheorem{exmp}{Example}[section]

\title{Heuristics for the Score-Constrained Bin-Packing Problem}
\author{Asyl L. Hawa \and Rhyd M. R. Lewis \and Jonathan M. Thompson}
\institute{School of Mathematics, Cardiff University, Senghennydd Road, Cardiff, UK, CF24 4AG}
\date{\today}

\begin{document}
\maketitle

\begin{abstract}
	This version finished: 13/01/2018
\end{abstract}


\section{Introduction}
\label{sec:intro}
The one-dimensional bin packing problem (BPP) is a combinatorial optimisation problem that has been widely researched and discussed due to its ability to model a variety of real-life sitations.
The BPP can be described as follows: given a set of $n$ items of varying sizes $w$, and a finite number of bins with equal capacity $c$, find the minimum number of bins required to contain all of the items. As shown by \cite{garey1979}, this problem is NP-hard, and therefore (under the assumption that $P \neq NP$) there is no known algorithm that is able to find an optimal solution in polynomial time for every instance of the BPP. Instead, heursitics can be used to find a near-optimal solution in a shorter amount of time. One example is the greedy heuristic known as first-fit, an online algorithm that places each item (in some arbitrary order) into the lowest-indexed bin. If there are no bins that can feasibly accommodate the item, the algorithm places the item into a new bin. A modified version of first-fit yields the well-known first-fit decreasing (FFD) algorithm (\citealp{eilon1971}), which initially sorts the items into non-decreasing order of size before allocating them into bins. As proven by \cite{dosa2007} \textcolor{RoyalBlue}{(proved bound is tight)}, the worst case for FFD is $\frac{11}{9}k + \frac{6}{9}$ bins, where $k$ is the optimal number of bins given by $\ceil{\frac{\sum_{i=1}^{n} w_{i}}{c}}$ \textcolor{RoyalBlue}{(..if the solution has this many bins then it is optimal,} (\citealp{korf2002})).

In 2004, \citeauthor{goulimis2004} brought to light an open-combinatorial problem related to the bin-packing problem. Corrugated boxes are formed in two stages: firstly, they are produced flat, then a device consistsing of knives mounted on a bar creates ``scores'' along specific lines on the flat boxes. These scores allow the boxes to be folded in predetermined places. The first stage of this process involves finding a feasible pattern of the flat boxes that minimises the amount of waste, and has been researched extensively. \textcolor{Plum}{It is classically solved by delayed column-generation} (\citealp{gilmore1961, gilmore1963}). However, the second step of this process requires an additional constraint.

The scoring knives, by design, cannot be placed too close together (around 70mm in industry), and as such have a ``minimum knife distance''. The problem then becomes a bin-packing problem with an additional constraint: find a feasible arrangement of boxes such that the sum of two adjacent score widths from different boxes is greater than or equal to the minimum knife distance. \textcolor{RoyalBlue}{We will call this constraint the minimum total score width constraint.}

\textcolor{OrangeRed}{Insert figure, two boxes annotated with score widths, score lines, box widths, knives mounted on bar showing minimum total score width constraint}

\textcolor{RoyalBlue}{Figure showing two boxes with score widths and the scoring knives mounted on a bar. Since the sum of the two adjacent score widths from different boxes is greater than or equal to 70mm, the minimum total width constraint has been met - this is a feasible alignment of these two boxes}

There exists a polynomial time algorithm that is able to recognise whether a particular instance of the problem is feasible or infeasible (\citealp{becker2010}) \textcolor{RoyalBlue}{(i.e. whether all boxes can be arranged in one line)}. However, the algorithm does not consider the lengths of each individual box, or the total length of the final arrangement of boxes. In industry, strips of materials are provided in fixed lengths, so, given a large number of boxes to pack, it may not be feasible to pack all of the boxes onto one strip. 

This leads us to our main problem, the score-constrained bin-packing problem: given a collection of boxes with varying widths and score widths on either side, and strips of a fixed length, find the minimum number of strips required to accommodate all of the boxes, such that the minimum total width constraint is met between all boxes on all strips. 

The remainder of this report will firstly, in Section \ref{sec:ffdapprox}, introduce the first-fit decreasing approximate algorithm and describe the modifications made to consider the minimum score width constraint. In Section \ref{sec:ffdexact}, we will present a new algorithm which consists of the modified first-fit decreasing algorithm detailed in the previous section combined with a polynomial-time algorithm which will be used to find feasible sub-solutions. A comparison of the two algorithms and an analysis of the results will be provided in Section \ref{sec:compresult}, and finally Section \ref{sec:conclusion} concludes the paper and proposes some potential directions for furture work.


\section{Formulating the problem}
We are given $n$ boxes that need to be packed onto strips. Each of these boxes have a positive width, and two score widths, one on either side of the box. Since these score widths are not necessarily equal, the boxes can be aligned in two ways: either in a ``regular'' alignment, where the smaller of the two score widths is on the left hand side of the box, or in a ``rotated'' alignment, where the box has been rotated by $180^{\circ}$ so that the larger score width is on the left-hand side of the box. Strips are packed left to right. We also have a finite number of strips of equal fixed lengths. Once a box has been placed onto a strip, the rightmost score on the strip is referred to as the ``free'' score. A box can only be placed onto a strip if it can fit onto the strip, i.e. the addition of the box to the strip doesn't exceed the length of the strip, and if the free score and one of the scores on the box meet the minimum total score width constraint, i.e. the sum of the free score width and one of the scores on the box is greater than or equal to the minimum total score width. Assume that no box has a width that is greater than the fixed length of the strips. \textcolor{RoyalBlue}{need to explain minimum total score width constraint with equation and example, used in the definition for SCBPP below.}


\begin{definition} 
	\label{defn:scbpp}
	Given $n$ boxes of varying widths $w(i)$ and score widths $w(i_1), w(i_2)$, and a finite number of strips $S_j$ with equal fixed lengths $l$, the Score-Constrained Bin-Packing Problem (SCBPP) consists of finding the minimum number of strips required to pack all of the boxes such that the minimum total score width constraint is met between every pair of boxes. \textcolor{RoyalBlue}{or: such that for every pair of boxes the sum of the adjacent score widths is greater than or equal to some fixed value $t$.}
\end{definition}



\section{FFD Using Approximate Algorithm}
\label{sec:ffdapprox}

\begin{algorithm}[H]
	\caption{\textcolor{OliveGreen}{Approximate First-Fit Decreasing Algorithm for the Score-Constrained Bin-Packing Problem}}
	\label{alg:ffdapprox}
	\begin{algorithmic}[1]
	\Require A list of boxes $\{i_1, i_2\}$ with widths $w(i), i = 1, 2, ..., n$, strips $S_j = \emptyset$ of fixed lengths $l$ and residual lengths $r(j) = l$, $j = 1, 2, ...,n$, minimum total score width $t$, and a set $\mathcal{S} = \emptyset$.
	\Ensure A feasible packing of all boxes onto a set of strips $\mathcal{S}$.
	\State Sort the boxes in order of non-increasing widths.
	\For{$i \gets 1 \To n$}
		\For{$j \gets 1 \To n$}
			\If{$S_j = \emptyset$}
				\State $S_j \gets \{i_1, i_2\}$
				\State $r(j) \gets r(j) - w(i)$
				\State $\mathcal{S} \gets \mathcal{S} \cup S_j$
				\Break
			\ElsIf{$S_j \neq \emptyset \land w(i) \leq r(j)$}
				\If{$w(f_j) + w(i_1) \geq t$}
					\State $S_j \gets S_j \cup \{i_1, i_2\}$
					\State $r(j) \gets r(j) - w(i)$
					\Break
				\ElsIf{$w(f_j) + w(i_2) \geq t$}
					\State $S_j \gets S_j \cup \{i_2, i_1\}$
					\State $r(j) \gets r(j) - w(i)$
					\Break
				\EndIf
			\EndIf
		\EndFor
	\EndFor
	\Return $\mathcal{S}$
	\end{algorithmic}	
\end{algorithm}

\textbf{\textcolor{Rhodamine}{DESCRIPTION OF FFDAPPROX}}

Initially, the algorithm arranges the boxes in non-increasing order of widths, breaking ties by choosing the box with the smallest score width. For each box $i$ in the sorted list, the algorithm find the lowest-indexed strip $S_j$ that is able to feasibly accommodate $i$. If $S_j$ is empty, $i$ is placed in a regular position onto $S_j$, which is then added to a set $\mathcal{S}$ that maintains a record of strips that are being packed. However, if $S_j$ is not empty, the algorithm checks the minimum total score width constraint with $i$ firstly in a regular position, then in a rotated position. If the constraint is met in either case, $i$ is placed accordingly onto $S_j$. If neither positions are able to fulfil the constraint, the algorithm moves onto the next strip $S_{j+1}$ and the process of attempting to pack $i$ is repeated. Once a box $i$ has been placed onto a strip, the algorithm moves onto the next box $i+1$, until all of the boxes have been packed. 



\section{FFD Using Exact Algorithm}
\label{sec:ffdexact}
Introduction

Explanation: Initialise

Let each box be represented by two vertices on a graph $G$, one vertex for each side of the box. We also introduce a ``dummy'' box that has two ``dominating scores'' with widths equal to the minimum total score width, $t$. This dummy box will be discarded at the end of the algorithm, and will not be included in the final feasible alignment. We now have $n$ boxes including the dummy box, and so $G$ will have $2n$ vertices.

We will refer two two vertices that represent either side of the same box as ``partners''. The boxes cannot be altered in any way, and so partners must be adjacent to one another. We therefore create a set of edges, $P$, consisting of edges between partners on $G$. \textcolor{RoyalBlue}{weighted edges $P$}.

Each vertex $v_i$ for $i = 1, 2, ..., 2n$ has a positive value $w(v)$ corresponding to the score width it represents. There is an edge between two vertices $u$ and $v$ on $G$ if and only if the vertices fulfil the \textit{minimum total score width constraint}, 

\begin{equation*}
	\label{eqn:mtswcg}
	w(u) + w(v) \geq t,
\end{equation*}

and $u$ and $v$ are not partners, i.e. $\{u, v\} \notin P$. The vertices $u$ and $v$ are referred to as \textcolor{OliveGreen}{``matched''}, and so their corresponding score widths can be feasibly aligned next to one another. We now have another set of edges, $Q$, consisting of edges between \textcolor{OliveGreen}{``matched''} vertices.  

In order to find a feasible alignment of the boxes, we must find an alternating Hamiltonian cycle on $G$. \textcolor{RoyalBlue}{turns into path when we remove dummy box.}

\begin{definition}
	\label{defn:althamcycle}
	Let $G(V, P\cup Q)$ be an undirected graph, where each edge is a member of one of two sets, $P$ or $Q$. $G$ has an \textit{alternating Hamiltonian cycle} if there exists a Hamiltonian cycle on $G$ such that the successive edges alternate between sets $P$ and $Q$. For example, if vertex $v_k$ in the Hamiltonian cycle is adjacent to vertex $v_{k-1}$ via an edge from $P$, then $v_k$ must be adjacent to vertex $v_{k+1}$ via an edge from $Q$, or vice versa. 
\end{definition}

\textcolor{OrangeRed}{Figure of graph $G$ with boxes, figure of aligned boxes with alternating Hamiltonian path underneath.}

\textcolor{RoyalBlue}{we now need to find a way of producing the alternating hamiltonian cycle on the graph $G$, if one exists.}

MTGMA/MIS

We need to obtain a set of matching edges $M \subset Q$ of cardinality $n$, such that the matching edges from $M$ and the edges from $P$ form an alternating Hamiltonain cycle as described in Definition \ref{defn:althamcycle}. We use a modified matching algorithm (\citealp{mahadev1994, becker2010}) that produces a suitable set of matching edges $M$. 

Initially, the vertices are sorted in order of non-decreasing values, and ties are broken arbitrarily. Starting from the smallest vertex, find the largest adjacent vertex (under $Q$? in the set $Q$? )that is not the current vertex's partner, and that has not yet been matched. Add this pair of vertices to the set $M$ that keeps a record of the matching edges, and proceed to the next vertex in the list. Continue until all vertices have been assessed. 

If a vertex $v_i$ is not adjacent to any other vertex except its partner, we can rematch previous vertices to provide $v_i$ with a feasible match, provided $v_i$ is not the first vertex in the sorted list, the previous vertex $v_{i-1}$ has a match (i.e. $v_{i-1} \in M$), and $v_{i-1}$ is adajcent to $v_i$'s partner. We then simply match $v_{i-1}$ with $v_i$'s partner, and match $v_i$ with $v_{i-1}$'s previous match. \textcolor{RoyalBlue}{max cardinality.}

If $|M| < n$, then there are not enough matching edges to form an alternating Hamiltonian cycle with the edges in $P$, and therefore there is no feasible arrangement of the boxes such that the minimum total score width constraint is not violated. However, if $|M| = n$, then the graph $G(V, P \cup M)$ is a 2-regular graph, where each vertex is adjacent to one vertex via an edge in $P$, (i.e. its partner), and another vertex via an edge in $M$ (i.e. its match). If $G$ consists of one alternating Hamiltonian cycle, then we simply remove the dominating vertices to obtain a feasible alignment of the boxes/feasible solution. If there are more than two cycles on $G$, we need to combine them into one alternating Hamiltonian cycle. \textcolor{RoyalBlue}{notation number of cycles.}

\textcolor{OrangeRed}{figure of graph $G$ showing edges between partners ($P$), edges in the set $M$, and faint lines to represent the rest of the edges $Q$.}


FCA/Patch

In order to combine multiple cycles into one, we need to replace some matching edges with feasible edges from the set $Q$ that connect vertices from different cycles. \textcolor{RoyalBlue}{The following algorithm is used to find these feasible edges}.

The algorithm initially creates a list of all matching edges in $M$, sorted in order of non-decreasing vertex values \textcolor{RoyalBlue}{(i.e. the edge which connects the smallest vertex to its match is first in the list) (do not include edges from partner swap)}. Starting from the first edge in the list, the algorithm searches through the list of edges until it finds an edge whose smaller vertex is adjacent via an edge in $Q$ to the larger vertex of the succeeding edge in the list, such that both edges are from different cycles on the graph $G$. If such a pair of edges has been found, the algorithm adds the first edge of the pair to a set $C_k$. The algorithm then continues to add all succeeding edges to $C_k$, provided that the smaller vertex of the current edge is adjacent to the larger vertex of the succeeding edge via an edge in $Q$, the succeeding edge does not belong to a cycle that is already represented by an edge in $C_k$, and the algorithm has not reached the final edge in the list. Once there are no more valid edges to add to $C_k$, the algorithm proceeds to the next edge in the list, and checked whether this edge is suitable for beginning a new set $C_{k+1}$. If the algorithm has already reached the penultimate or final edge in the list, the algorithm terminates.

If the algorithm does not produce a single set $C$, there are no edges available to connect the cycles together, and therefore no feasible alignment of the boxes in this problem instance exists. \textcolor{RoyalBlue}{also if the set $C$ doesn't include an edge from every cycle, or multiple sets $C$ that don't cover all cycles, infeasible.}

If there exists a set $C_k$ such that $|C_k| = \textcolor{OliveGreen}{number of cycles}$, then the cycles can be connected as follows: starting from the first edge in $C_k$, connect the smaller vertex to the larger vertex of the second edge in the set. Then, connect the smaller vertex of the second edge to the larger vertex of the third edge. Continue in this manner until the last edge in $C_k$ is reached, and connect the lower vertex of this final edge to the larger vertex of the first edge. These new matching edges will join the distinct cycles together to form one alternating Hamiltonian cycle. The edges in the set $C_k$ can be removed from the graph $G$

\textcolor{RoyalBlue}{Multiple $C_k$ cycles required.}

\begin{algorithm}[H]
	\caption{\textcolor{OliveGreen}{Exact First-Fit Decreasing Algorithm for the Score-Constrained Bin-Packing Problem}}
	\label{alg:ffdexact}
	\begin{algorithmic}[1]
	\Require A list of boxes $\{i_1, i_2\}$ with widths $w(i), i = 1, 2, ..., n$, strips $S_j = \emptyset$ of fixed lengths $l$ and residual lengths $r(j) = l$, $j = 1, 2, ...,n$, and a set $\mathcal{S} = \emptyset$.
	\Ensure A feasible packing of all boxes onto a set of strips $\mathcal{S}$.
	\State Sort the boxes in order of non-increasing widths
	\For{$i \gets 1 \To n$}
		\For{$j \gets 1 \To n$}
			\If{$S_j = \emptyset$}
				\State $S_j \gets \{i_1, i_2\}$
				\State $r(j) \gets r(j) - w(i)$
				\State $\mathcal{S} \gets \mathcal{S} \cup S_j$
				\Break
			\ElsIf{$S_j \neq \emptyset \land w(i) \leq r(j)$}
				\State \textcolor{RoyalBlue}{Run} \textcolor{OrangeRed}{``MBAHRA''} \textcolor{RoyalBlue}{on boxes in $S_j$ and box $i$}
				%\Return \textcolor{OrangeRed}{found}
				\If{feasible alignment found}
				%\If{found}
					\State $S_j \gets$ \textcolor{RoyalBlue}{new alignment found by} \textcolor{OrangeRed}{``MBAHRA''}
					\State $r(j) \gets r(j) - w(i)$
					\Break
				\EndIf
			\EndIf
		\EndFor
	\EndFor
	\Return $\mathcal{S}$
	\end{algorithmic}	
\end{algorithm}

\textbf{\textcolor{ProcessBlue}{DESCRIPTION OF FFDEXACT}}

\textcolor{RoyalBlue}{In a similar fashion/manner to ``FFD approx'' (rename)} The algorithm starts by sorting the boxes in order of non-increasing widths, with ties broken by choosing the box with the smallest score width. Starting with the first box, the algorithm finds the lowest-indexed strip $S_j$ that can feasibly accommodate the current box $i$. If $S_j$ is empty, box $i$ is placed onto $S_j$ in a ``regular'' position \textcolor{RoyalBlue}{(i.e. smallest score width on the LHS of the box)}. This strip is added to the set $\mathcal{S}$ that keeps track of the strips being used to form a solution, and the algorithm proceeds onto the next box $i+1$ in the sorted list.

If $S_j$ is not empty \textcolor{RoyalBlue}{(i.e contains at least one box, $|S_j| > 0$)}, then \textcolor{OrangeRed}{MBAHRA} is performed on all of the boxes in $S_j$ and the current box $i$. If a feasible alignment of the boxes has been found, the current boxes on $S_j$ are replaced with the new alignment, and the algorithm proceeds onto the next box $i+1$. If \textcolor{OrangeRed}{MBAHRA} is unable to find a feasible alignment of the boxes, the algorithm moves onto the next strip $S_{j+1}$ and the process is repeated. The algorithm continues until all of the boxes have been packed. \textcolor{RoyalBlue}{Complexity}

\section{Comparison and Results}
\label{sec:compresult}


\section{Conclusion}
\label{sec:conclusion}

\section{References}
\cite{becker2010}, \cite{becker2015}, \cite{coffman1978}, \cite{dosa2007}, \cite{eilon1971}, \cite{garey1979}, \cite{gilmore1961}, \cite{gilmore1963}, \cite{goulimis2004}, \cite{johnson1974}, \cite{karp1972}, \cite{korf2002}, \cite{lewis2009}, \cite{lewis2017}, \cite{lewis2011}, \cite{mahadev1994}, \cite{mahadev1995}, \cite{martello1990a}, \cite{martello1990b}

\section{Layout}
Introduction:
\begin{itemize}
	\item One-dimensional BPP, NP-Hard.
	\item Goulimis packing boxes onto \textit{one} strip, where boxes have score constraints.
	\item Describe boxes, scores, score widths, minimum total score width constraint, scoring knives on bar etc. (include figure(s)).
	\item Becker has created a polynomial time algorithm for this problem (again, only on \textit{one} strip).
	\item How we adapt this problem: we take into account the width of each individual box (added widths to the boxes, not included by Becker), multiple strips, fixed lengths, if there was an instance with many boxes it's unlikely they would all fit onto one strip, strips come in fixed lengths that cannot be changed.
	\item Our problem: pack the boxes onto multiple strips of fixed lengths while meeting the minimum total score width constraint, not exceeding the capacity of the strips, and trying to minimise the number of strips used (perhaps mention minimising waste per strip?).
	\item Are there any similar problems? Who has researched this problem/similar problem, and using what methods? \cite{lewis2017} trapezoids paper. 
	\item Explain \textit{why this problem is important} (see above, strips come in fixed lengths, becker's algorithm is rendered useless if the total width of all boxes exceeds the length of one strip).
	\item Outline rest of paper.	
\end{itemize}

Lit Review
\begin{itemize}
	\item Solving the one-dimensional bpp using exact methods and heuristics.
	\item Exact methods - \cite{martello1990a} branch and bound based exact algorithm MTP, Carvalho 1999 column generation and branch-and-bound, \cite{eilon1971} depth-first search enumerative algorithm, only solve small size problem instances.
	\item \cite{garey1979} \textcolor{RoyalBlue}{NP-Hard, and therefore (under the assumption that $P \neq NP$) there is no known algorithm that is able to find an optimal solution in polynomial time for every instance of the BPP.}
	\item \textcolor{RoyalBlue}{Instead, heuristics can be used to find a near optimal solution in a shorter amount of time}.
	\item discuss heuristics.
	\item Explain FF and FFD, online/offline \cite{eilon1971}, \cite{dosa2007} tight bound, optimal number of bins $k$ \cite{korf2002}, complexity $O(n \log n)$ \cite{coffman1984} (where n is the number of items to be packed).
	\item quickly produce a feasible solution, but may not be optimal.
	\item Use heuristics in this paper to find feasible solutions for the problem, why?
\end{itemize}

Notation and Formulating the problem
\begin{itemize}
	\item $n$ boxes, $i = 1, 2, ..., n$, same height, different widths.
	\item Each box $i$ has a total width $w(i)$.
	\item Each box $i$ has two scores, $i_1$ (the smaller score), and $i_2$ (the larger score), and widths $w(i_1), w(i_2)$, such that $w(i_1) \leq w(i_2)$.
	\item Score widths not necessarily equal, therefore box has two different positions.
	\item $\{i_1, i_2\}$ represents ``regular'' box $i$: $i_1$ on LHS of box.
	\item $\{i_2, i_1\}$ represents ``rotated'' box $i$: $i_2$ on LHS of box (box rotated by $180^{\circ}$).
	\item Boxes will be placed side by side.
	\item All box widths $w(i) > 0$ and all score widths $w(i_1), w(i_2) > 0$ $\forall$ $i = 1, 2, ..., n$.
	\item Finite number of strips $S_j$, $j = 1, 2, ...$.
	\item All strips $S_j$ have equal fixed length $l > 0$ $\forall$ $j = 1, 2, ...$.
	\item All strips $S_j$ have a residual length $r(j)$, initially $r(j) = l$ $\forall$ $j = 1,2, ...$, which reduces as boxes are added to each strip.
	\item Assumption: all box widths $w(i) \leq l, i = 1, 2, ..., n$
	\item Initially, $S_j = \emptyset$ $\forall$ $j = 1, 2, ...$.
	\item For all strips $S_j \neq \emptyset$, the score on the right hand side of the last box in the strip is called the ``free score'', denoted $f_j$, with width $w(f_j)$.
	\item The sum of two adjacent score widths must fulfil the minimum total score width constraint ... $t$ is the minimum total score width.
	\item The minimum total score width constraint must be fulfilled in order for the scoring knives to be able to score the boxes correctly.
	\item The \textit{minimum total score width} is denoted by $t, t > 0$ \textcolor{RoyalBlue}{usually 70mm in industry}. \textcolor{RoyalBlue}{The \textit{minimum total score width constraint} is $w(u_p) + w(v_q) \geq t$, $u \neq v, p, q = 1 or 2$.}
	\item Strips packed from left to right.
	\item $l > 0, w(i) > 0, w(i) \leq l$
	\item Need to pack the boxes onto the strips so mtswc is met and least number of strips used.
	\item Formal definition of Score-Constrained Bin Packing Problem
\end{itemize}

FFD Approx
\begin{itemize}
	\item Explain FFD modified for scored boxes (basic).
	\item Free score $f_j$, residual length $r(j)$.
	\item Figure: partially packed strip showing first box in regular position, free score, minimum total score width constraint met between boxes, residual length.
	\item Pseudocode
	\item Description of algorithm
	\item Complexity, discuss how algorithm only checks if current box meets constraint with free score on final box, boxes are not able to be rearranged, even though there may be another arrangement of the boxes that is feasible (lead on to next algorithm FFD using exact algorithm).	
\end{itemize}

FFD Using Exact Algorithm
\begin{itemize}
	\item Combine FFD with exact algorithm to find a feasible solution to the SCBPP
	\item Unlike FFD approx, where attempts are made to place each box at the end of a strip, this algorithm is able to rearrange the order of the boxes in a strip to find a feasible arrangement.
	\item Can find a solution for a set of boxes on a strip when it may not have possible to find one in ffdapprox
	\item Describe Becker algorithm, using graph theory, Hamiltonian path/cycle
	\item Notation for Exact algorithm:
	\item $n$ boxes including dummy box.
	\item $2n$ vertices including dominating vertices.
	\item Graph $G$ is an undirected graph with $2n$ vertices $v_i, i = 1, 2, ...,2n$.
	\item Vertices are weighted, $w(v_i)$.
	\item Vertices that represent either side of same box referred to as ``partners''
	\item $P$ contains the set of edges between partners, fixed, $|P| = n$.
	\item Edges in $P$ are weighted, corresponding to box width.
	\item $Q$ contains the set of edges between vertices that meet the minimum total score width constraint, excluding partners (these edges are not weighted).
	\item $G(V, P \cup Q)$ - each edge on $G$ is either in $P$ or $Q$, not both.
	\item $t$ still represents the minimum total score width.
	\item $M$ is the set of matching edges, $M \subset Q$, edges chosen using the matching algorithm (need $|M| = n$).
	\item number of cycles on $G$ after MTGMA =  ?
	\item $C_k$: sets of edges for connecting cycles.
	\item Section: Becker algorithm
	\item Combine alg with ffd
	\item Pseudocode
	\item Description of algorithm
	\item Complexity
\end{itemize}

Comparison/Results

Conclusion












\bibliographystyle{dcu}
\bibliography{includes/bibliography}

\end{document}

