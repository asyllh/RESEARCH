\documentclass{llncs}

%---- PREAMBLE ----%

% Allows separate files to be used in the main file
\usepackage{subfiles}

% Algorithms
\usepackage{algorithm}
\usepackage[noend]{algpseudocode}
\renewcommand{\algorithmicrequire}{\textbf{Input:}}
\renewcommand{\algorithmicensure}{\textbf{Output:}}
\newcommand{\IndStatex}[1][1]{\Statex\hspace{6mm}}
\newcommand{\IndIndStatex}[1][1]{\Statex\hspace{12mm}}
\newcommand{\IndIndIndStatex}[1][1]{\Statex\hspace{18mm}}

% Contains advanced math extensions
\usepackage{amsmath}

% Introduces the *proof* environment and the \theoremstyle command
\usepackage{amsthm}

% Adds new symbols to be used in math mode, e.g. \mathbb
\usepackage{amssymb}

% To declare multiple authors
\usepackage{authblk}

% Provides extra comands as well as optimisation for producing tables
\usepackage{booktabs}

% Allows customisation of appearance and placements for figures/tables etc.
\usepackage{caption}

% Adds support for arbitrarily-deep nested lists
\usepackage{enumitem}

% Improves the interface for defining floating objects such as figures/tables
\usepackage{float}

\usepackage{fullpage}

% For easy management of document margins and the document page size
%\usepackage{geometry}

% Allows insertion of graphic files within a document
\usepackage{graphicx}

% Manage links within the document or to any URL when you compile in PDF
\usepackage{hyperref} 

% Successor of amsmath
\usepackage{mathtools}

% No indentation, space between paragraphs
\usepackage{parskip}

%Include standalone .tex files
\usepackage{standalone}

% Define multiple floats (figures/tables) within one environment with individual captions 1a, 1b etc
\usepackage{subcaption}

\usepackage{tikz}

\usepackage{wrapfig}

%Theorem style
\theoremstyle{plain}% default
\newtheorem{theorem}{Theorem}[section]
\newtheorem{corollary}{Corollary}[theorem]

\theoremstyle{definition}
\newtheorem{defn}{Definition}[section]
\newtheorem{proposition}{Proposition}[defn]
\newtheorem{exmp}{Example}[section]

\title{Notes for IWOCA Paper}
\author{Asyl L. Hawa}

\begin{document}
\maketitle

\begin{abstract}
	Abstract
\end{abstract}

\section{Introduction}

\begin{itemize}
	\item \textcolor{red}{original BPP}
	\item Definition of problem: packing items of different sizes into a finite number of bins/containers that have a fixed size such that the number of bins/containers required is minimised.
	\item Combinatorial NP-hard problem (computational complexity) Gary and Johnson 1979
	\item NP-Complete (decision problem)
	\item Heuristics
	\item FF and FFD
	\item FF can produce a feasible solution quickly, but the solution may not be optimal O(nln n), n is the number of items that need to be packed.
	\item ``There exists at least one ordering of items that allows FF to produce an optimal solution'' (wiki) Rhyd Lewis 2009
	\item ``bound for FFD is tight'' (wiki)
	\item 2015 Hoberg and Rothvoss proposed and proofed a new complexity for the algorithm FF (wiki)
	\item Exact algorithms, Martello and Toth (MTP), Bin Completion algorithm Korf 2002, Schreiber and Korf (2013) (wiki)
	\item \textcolor{blue}{Although there are exact solutions available for the BPP} (martello knapsack MTP 1989 (1990b) ``is based on a first-fit decreasing'' branching strategy and is the best existing algorithm for optimal packing (KOrf2002)'', hung and brown 1978 ``a branch-and-bound algorithm'' paper title:``an algorithm for a class of loading problems'', eilon christofides 1971 ``a depth-first numerative algorithm'' paper title: ``the loadings problem'', schreiber and korf 2002 bin completion algorithm) \textcolor{blue}{they are only able to find solutions to problems that have a small input size.}
	\item ``Hence, exact solution techniques are bound to work well for small to medium sized problem instances only, and real world sized problems including up to thousands of items have to be solved heuristically.''
	
	\item \textcolor{red}{Score constraints}	
	\item Goulimis 2004, score constraints for packing and folding
	\item Explain knives for scoring
	\item Picture of box with score widths, score lines
	\item All boxes have same height, different widths, different score widths
	\item ``Threshold'' ``minimum knife distance'' ``minimum score separation constraint'' ``minimum score line distance'', ``minimum score line constraint'' ``minimum total width constraint'' 70mm industry
	\item Becker alg exact, P, but without packing, all boxes on one strip
	\item \textcolor{red}{Multiple strips}
	\item Material may not come in long sizes, instead short sizes of fixed length
	\item find optimal arrangement of boxes on strips while mimimising waste and number of strips required
	
	\item Formal statement of BPP with score constraint called the Score Constrained Bin packing problem: given ...
	
	\item layout of paper
\end{itemize}

\begin{enumerate}
	\item Goulimis (2004)
	\item boxes are scored in specfici places to make folding easier 
	\item only consider outermost scores, although box may have many 
	\item machine explanation, scoring knives cannot be placed too close together
	\item therefore we have a score constraint/minimum knife distance
	\item strips of material of a fixed length
	\item minimise waste whilst adhering to the score constraints
	\item feasible arrangement of boxes
	\item boxes have same heights, different widths
	\item therefore have a strip packing/bpp with an additional constraint
	\item explain rest of paper
\end{enumerate}

\begin{itemize}
	\item FF if online, FFD is offline
	\item FF and FFD can produce a feasible solution quickly, but the solution may not be optimal O(nln n), where n is the number of items that need to be packed
	\item Exact algorithms, Martello and Toth (MTP), Bin Completion algorithm Korf 2002, Schreiber and Korf (2013) (wiki)
	\item \textcolor{RoyalBlue}{Although there are exact solutions available for the BPP} (martello knapsack MTP 1989 (1990b) \textcolor{Plum}{``is based on a first-fit decreasing'' branching strategy and is the best existing algorithm for optimal packing (Korf2002)'', hung and brown 1978 ``a branch-and-bound algorithm'' paper title:``an algorithm for a class of loading problems'', eilon christofides 1971 ``a depth-first numerative algorithm'' paper title: ``the loadings problem'', schreiber and korf 2002 bin completion algorithm)} \textcolor{RoyalBlue}{they are only able to find solutions to problems that have a small input size.}
	\item \textcolor{Plum}{``Hence, exact solution techniques are bound to work well for small to medium sized problem instances only, and real world sized problems including up to thousands of items have to be solved heuristically.''}
	\item \textcolor{Plum}{``There exists at least one ordering of items that allows FF to produce an optimal solution'' (wiki) Rhyd Lewis 2009}
	\item All boxes have same height, different widths, different score widths
	\item Recognition algorithm Becker
	\item \textcolor{RoyalBlue}{Formal statement of BPP with score constraint called the Score Constrained Bin packing problem (SCBPP): given}
	\item dicuss other paper/articles with this problem, i.e. rhyd 2017	
\end{itemize}

\begin{itemize}
	\item $n$ boxes need to be packed, all have same height but different widths $w(i)$.
	\item Each box $i$ has two scores $i_1, i_2$, one on either side of the box.
	\item Each of these scores has a width $w(i_1) \leq w(i_2)$.
	\item Score widths not necessarily equal, therefore box has two different positions.
	\item Regular position, smaller score on LHS, denoted $\{i_1, i_2\}$.
	\item Rotated position, smaller score on RHS, denoted $\{i_2, i_1\}$ (rotated $180^{\circ}$).
	\item Boxes will be placed side by side.
	\item The sum of two adjacent score widths must fulfil the minimum total score width constraint ... $t$ is the minimum total score width.
	\item The minimum total score width constraint must be fulfilled in order for the scoring knives to be able to score the boxes correctly.
	\item There are a finite number of strips $S_j$, with equal fixed lengths $l$.
	\item Need to pack the boxes onto the strips so mtswc is met and least number of strips used.
	\item $l > 0, w(i) > 0, w(i) \leq l$
\end{itemize}

Notation
\begin{itemize}
	\item $n$ boxes, $i = 1, 2, ..., n$.
	\item Each box $i$ has a total width $w(i)$.
	\item Each box $i$ has two scores, $i_1$ (the smaller score), and $i_2$ (the larger score), and widths $w(i_1), w(i_2)$, such that $w(i_1) \leq w(i_2)$.
	\item $\{i_1, i_2\}$ represents ``regular'' box $i$: $i_1$ on LHS of box.
	\item $\{i_2, i_1\}$ represents ``rotated'' box $i$: $i_2$ on LHS of box (box rotated by $180^{\circ}$).
	\item All box widths $w(i) > 0$ and all score widths $w(i_1), w(i_2) > 0$ $\forall$ $i = 1, 2, ..., n$.
	\item Finite number of strips $S_j$, $j = 1, 2, ...$.
	\item All strips $S_j$ have equal fixed length $l > 0$ $\forall$ $j = 1, 2, ...$.
	\item All strips $S_j$ have a residual length $r(j)$, initially $r(j) = l$ $\forall$ $j = 1,2, ...$, which reduces as boxes are added to each strip.
	\item Assumption: all box widths $w(i) \leq l, i = 1, 2, ..., n$
	\item Initially, $S_j = \emptyset$ $\forall$ $j = 1, 2, ...$.
	\item For all strips $S_j \neq \emptyset$, the score on the right hand side of the last box in the strip is called the ``free score'', denoted $f_j$, with width $w(f_j)$.
	\item The \textit{minimum total score width} is denoted by $t, t > 0$ \textcolor{RoyalBlue}{usually 70mm in industry}. \textcolor{RoyalBlue}{The \textit{minimum total score width constraint} is $w(u_p) + w(v_q) \geq t$, $u \neq v, p, q = 1 or 2$.}
	\item Strips packed from left to right.
	
	\item -------------------------
	
	\item Notation for Exact algorithm
	\item $n$ boxes including dummy box.
	\item $2n$ vertices including dominating vertices.
	\item Graph $G$ is an undirected graph with $2n$ vertices $v_i, i = 1, 2, ...,2n$.
	\item Vertices are weighted, $w(v_i)$.
	\item Vertices that represent either side of same box referred to as ``partners''
	\item $P$ contains the set of edges between partners, fixed, $|P| = n$.
	\item Edges in $P$ are weighted, corresponding to box width.
	\item $Q$ contains the set of edges between vertices that meet the minimum total score width constraint, excluding partners (these edges are not weighted).
	\item $G(V, P \cup Q)$ - each edge on $G$ is either in $P$ or $Q$, not both.
	\item $t$ still represents the minimum total score width.
	\item $M$ is the set of matching edges, $M \subset Q$, edges chosen using the matching algorithm (need $|M| = n$).
	\item number of cycles on $G$ after MTGMA =  ?
	\item $C_k$: sets of edges for connecting cycles.
	
\end{itemize}


\section{FFD with approximate heuristic}
\begin{enumerate}
	\item background on FF/FFD
	\item explain how FFD operates without score constraints (simple)
	\item complexity/lower bound
	\item explain FFD with score constraint (algorithm)
	\item pros and cons: speed, however there may be a solution that exists that cannot be found due to boxes being fixed in place
\end{enumerate}

\begin{itemize}
	\item discuss heuristics
	\item \textcolor{Plum}{``Heuristics are approximate solution techniques that typically provide solutions in the least amount of time to the detriment of the solutions quality''}
	\item \textcolor{Plum}{``Exact methods find a best possible packing, but are slow and may be unable to provide solutions to large or reastically sized problem instances within reasonable time''}
	\item \textcolor{Plum}{``Consider a set of $n$ items. In the context of bin packing problems, heuristics consider these items one-by-one and attempt to pack them into a bin so as to achieve a near-optimal solution without the guarantee that a solution is optimal.''}
	\item \textcolor{Plum}{``Exact packing methods are often called deterministic methods and are guarantees to find an optimal solution to a problem (given sufficient time and a feasible region) while heuristics attempt to find optimal solutions, but are not guaranteed to do so.''}
	\item \textcolor{Plum}{``Exact methods may often be too slow to solve large problem instances, but may typically be used to solve smaller ones.''}
	\item \textcolor{Plum}{``Packing heuristics follow a fixed set of rules to pack items in such a manner as to find good, feasible (but not necessarily optimal) solutions to the bin packing problem within as short a time span as possible.''}
	\item offline algorithm, whole problem data is provided/required at the start.
	\item online, doesn't consider future events/items, data is processed in the order that it is input into the algorithm, entire input is not required to be available from the start
	\item \textcolor{Plum}{``first-fit decreasing (FFD) algorithm for one-dimensional bin packing by johnson et al (1974, worst-case performance bounds for simple one-dimensional packing algorithms)''}
	\item FF is online, FFD is offline
	\item FF and FFD can produce a feasible solution quickly, but the solution may not be optimal, O(nln n), where n is the number of items that need to be packed
	\item \textcolor{Plum}{who stated that ffd is O(nlog n)?}
	\item \textcolor{Plum}{Coffman et al 1984 heuristics survey}
	\item \textcolor{RoyalBlue}{Describe how the ffdapprox algorithm works}
	\item \textcolor{OrangeRed}{figure of packed boxes on strips}
	\item previously placed boxes are fixed, cannot be rearranged
	\item complexity of modified FFD for SCBPP
	\item background information on FFD
	\item works in the same way as FFD, but with an additional step - each box must meet the minimum total score width constraint, so if a box fits onto a strip, we start by checking if the smallest of the two score widths on the box, and the free score width on the end of the current strip, meet the mtswc. If not, then we rotate the box $180^{\circ}$ and check again if the mtswc is met. If neither score width on the box meets the mtswc with the free score width, then we move onto the next strip and try again.
	\item pack from left to right
	\item boxes can be aligned in two ways, with smaller score on LHS, or rotated $180^{\circ}$ so the smaller score is on the LHS
	\item Karp complexity
	\item \textcolor{OrangeRed}{figure of one packed strip indicating free score end $f_j$}
	\item pros and cons: speed, however there may be a solution that exists that cannot be found due to boxes being fixed in place
	\item In this algorithm, the boxes are sorted by non-increasing widths prior to packing.
	\item If a box does not fit onto any existing strips, a new strip is initialised.
	\item Initially, pack the largest (widest/box with the largest width/first box in the sorted list of boxes) box into the first strip.
	\item \textcolor{Plum}{``Simchi-Levi (1994, ``New worst-case results for the bpp'') has shown that the FFD and the BFD heuristics have an absolute performance ratio of 1.5, and that this value is the best possible for the BPP, unless $P = NP$.''}
	\item We use $\{i_1, i_2\}$ or $\{i_2, i_1\}$ to represent the box, rather than just $i$, because the feasibility of the alignment of boxes on the strip depends on which way the box is placed, regularly or rotated.
\end{itemize}

\section{FFD with exact algorithm}
\begin{enumerate}
	\item extension of the above, however we include an algorithm that can find a feasible alignment of boxes with score constraints if one exists
	\item if a box can fit onto a strip, instead of attempting to place on the end, we apply this algorithm
	\item therefore unlike the heuristic above, boxes are not fixed in place, all boxes can be reordered
	\item then go on to explain mbahra + output of solution
	\item pros and cons? polynomial time, however it means that it cannot be done continuously, i.e. the heuristic can be applied during scoring, but with the exact alg, a full solution must be found before any scoring can occur, as boxes can be moved around
\end{enumerate}

Introduction
\begin{itemize}
	\item This alg will be combining a well known heuristic with an exact algorithm to find a feasible solution to the SCBPP
	\item Becker 2010 recognicity algorithm in polynomial time
	\item doesn't take into account individual widths of boxes
	\item only one strip/align all boxes into one row
	\item in industry, strips have a finite length
	\item algorithm also doesn't output a full solution, only whether there exists a feasible solution or not
	\item ffd but including an algorithm that can find a feasible aligment of boxes with score constraints if one exists
	\item unlike ffd, where boxes are placed onto the end of each strip, the boxes are not fixed in place, all boxes can be reordered.
	\item Can find a solution for a set of boxes on a strip when it may not have been possible to find one in ffdapprox due to only the last box being checked.
	\item try to find a hamiltonian path 
	\item complexity $O(n^2), O(n)$ to construct a solution
	\item if the algorithm cannot find a feasible alignment of the boxes including the current box, move onto next strip and try again.
\end{itemize}

\begin{definition}
	% Alternating Hamiltonian Path/cycle
	Let $G(V,P\cup E)$ be an undirected graph, where $P \cap E = \emptyset$. Then $G$ has an alternating Hamiltonian path (or cycle) if there exists a Hamiltonian path (or cycle) on $G$ such that the successive edges are members of different edge sets, *i.e. if vertex $v_k$ is adjacent to vertex $v_{k-1}$ via an edge from $P$, then $v_k$ must be adjacent to vertex $v_{k+1}$ via an edge from $E$, or vice versa. $G$ is said to be alternating Hamiltonian if and only if there exists an alternating Hamiltonian cycle on $G$. (OR *i.e. if $\{v_{k-1}, v_k\} \in P$, then $\{v_k, v_{k+1}\} \in E$, or vice versa.)
\end{definition}





\subsection{Intro}
\begin{enumerate}
	\item Notation
	\item Creating two sets of edges
	\item Alternating Hamiltonian Path/cycle
\end{enumerate}
\begin{itemize}
	\item $n$ boxes, each box has two score lines, one on each side (mates)
	\item Score width - distance between edge of box and score line
	\item scores come in pairs
	\item let each box be represented by two vertices, one vertex for each score
	\item vertex weight equivalent to score width (mm)
	\item if two scores are part of the same box, join corresponding vertices together with an edge
	\item call this set of edges $P$, where $|P| = n$.
	\item edges in $P$ are weighted, equivalent to box width
	\item threshold graph: if sum of two vertices $\geq$ threshold, and vertices are not connected via an edge from $P$, i.e. $\{u, v\} \notin P$, then join vertices together via an edge
	\item call this set of edges $T$
	\item two sets of edges, $P$ and $T$
	\item add dominating vertices
	\item we want to find an alternating hamiltonian path, edges alternating between set $P$ and set $T$.
	\item -----------------------
	\item each box has same height, varying width (introduction?)
	\item vertices have values corresponding to the score width
	\item two vertices that represent either side of the same box are called ``mates''
	\item add dominating vertices that are mates with value = mkd, removed at the end
	\item so graph has $2n + 2$ vertices
	\item mates must be adjacent to one another, so we have a set of edges $P$ that consists of edges between mates
	\item each of these edges are weighted, value corresponds to boxwidth
	\item seconds set of edges $T$: if the sum of two score widths from different boxes is greater than or equal to the mkd, the corresponding vertices will be connected with an edge 
	\item called these matched
	\item or if the sum of two vertices that are not adjacent via an edge from $P$ is $\geq$ mkd, connect via an edge
	\item picture threshold graph colours or dots?	
\end{itemize}


--------------------------------\\
Let each box be represented by two vertices on a graph $G$, one vertex for each side of the box. Each vertex is assigned a value corresponding to the width of the score it represents. We refer to two vertices that represent either side of the same box as ``mates''. The boxes cannot be altered in anyway, and so vertices that are mates must be adjacent to one another. Thus, we have a set of edges $P$ that consists of edges between mates.

As described in the introduction, let $\tau \in \mathbb{R}_{+}$ be the minimum scoring distance. If the sum of the values of two vertices $\{u, v\}$ is greater than or equal to $\tau$, and $\{u,v\} \notin P$, then we can connect these two vertices with an edge. Hence we have another set of edges $T$ consisting of edges between vertices that could ?  

\subsection{MTGMA}
\begin{enumerate}
	\item explain max cardinality matching algorithm excluding edges from $P$
	\item swap of mates
	\item if $|M| = n$, continue, else fail, no feasible arrangement of the boxes exists
\end{enumerate}
\begin{itemize}
	\item need a suitable matching $M\subset E$ s.t. $|M| = n-1$ CHECK THIS NOW WE HAVE DOMINATING VERTICES
	\item maximum cardinality matching algorithm for threshold graphs (Mahadev and Peled 1994)
	\item modify algorithm so that duplicate edges from mates are deleted, only consider edges from matching
	\item explain swap of mates??
\end{itemize}


There is a simple maximum cardinality matching algorithm (Mahadev and Peled, 1994) that produced a matching ?? containing suitable edges from $T$.

Initially, the vertices are sorted in order of non-decreasing weights, with ties broken arbitrarily. Starting from the vertex with the smallest weight, attempt to match it with any vertex in its neighbourhood that has not yet been matched. If such a vertex is available, add the current vertex and its match to a set $I$ that keeps track of the matched vertices. Succesively proceed to the next largest vertex until all vertices have been assessed.



\subsection{MIS}
\begin{enumerate}
	\item new graph, each vertex has degree 2 (connected to one vertex via edge from $P$ and one vertex via edge from $M$)
	\item either one cycle, remove dominating vertices
	\item 2+ cycles, need to connect all cycles together
\end{enumerate}


\subsection{FCA}
\begin{enumerate}
	\item find pairs of vertices in each cycle connected via a matching edge that can be broken and rematched
	\item lower vertex adj to higher vertex of other cycle's edge
	\item connect cycles together, remove dominating vertices
	\item if cycles cannot be connected, then no feasible arrangement of all boxes exists
\end{enumerate}

\subsection{Other}
\begin{enumerate}
	\item benefits of this alg wrt FFD/BPP?
\end{enumerate}

DESCRIPTION OF FFD EXACT
\begin{itemize}
	\item Start with boxes sorted in order of non-increasing widths, breaking ties by choosing the box with the smallest score width.
	\item Place the first box onto the first strip, such that the smaller of the two score widths is on the leftmost side of the strip.
	\item Attempt to fit each box $i, i = 2,..., n$ onto the lowest-indexed strip possible that already contains a box, i.e. the strip is not empty.
	\item If a box is able to feasibly fit onto a strip, apply the \textcolor{OliveGreen}{algorithm} on all of the boxes on the strip plus the current box $i$.
	\item If a feasible alignment of the boxes has been found, replace the current alignment of boxes on the strip with the new alignment, which includes box $i$ (and move onto the next box).
	\item If no feasible alignment of the boxes is found, we move onto the next lowest-indexed strip and repeat.
	\item If a box cannot fit into any of the used strips, i.e. any of the strips that already contain a box/any of the open strips, we then place the box onto a new, unused strip, such that the smaller of the two score widths on the box is on the leftmost side of the strip, then move onto the next box.
	\item Continue untill all boxes have been placed onto a strip/until all boxes have been packed.
\end{itemize}


\section{Experiments/Comparison}
\begin{itemize}
	\item \textcolor{Plum}{``Dowsland and Dowsland (1992, ``Packing Problems'') comment that while average or worst-case performance bounds are useful guidelines, it is best to determine an algorithm's usefuless by testing it on data sets typical to the intended problem. Repositories are available on the internet where benchamrk data sets for packing problems are stored for the purpose of evaluating algorithms.''}
\end{itemize}


\section{Conclusion}

\section{Layout}
Introduction:
\begin{itemize}
	\item One-dimensional BPP, NP-Hard.
	\item Goulimis packing boxes onto \textit{one} strip, where boxes have score constraints.
	\item Describe boxes, scores, score widths, minimum total score width constraint, scoring knives on bar etc. (include figure(s)).
	\item Becker has created a polynomial time algorithm for this problem (again, only on \textit{one} strip).
	\item How we adapt this problem: we take into account the width of each individual box (added widths to the boxes, not included by Becker), multiple strips, fixed lengths, if there was an instance with many boxes it's unlikely they would all fit onto one strip, strips come in fixed lengths that cannot be changed.
	\item Our problem: pack the boxes onto multiple strips of fixed lengths while meeting the minimum total score width constraint, not exceeding the capacity of the strips, and trying to minimise the number of strips used (perhaps mention minimising waste per strip?).
	\item Are there any similar problems? Who has researched this problem/similar problem, and using what methods? \cite{lewis2017} trapezoids paper. 
	\item Explain \textit{why this problem is important} (see above, strips come in fixed lengths, becker's algorithm is rendered useless if the total width of all boxes exceeds the length of one strip).
	\item Outline rest of paper.	
\end{itemize}

Lit Review
\begin{itemize}
	\item Solving the one-dimensional bpp using exact methods and heuristics.
	\item Exact methods - \cite{martello1990a} branch and bound based exact algorithm MTP, Carvalho 1999 column generation and branch-and-bound, \cite{eilon1971} depth-first search enumerative algorithm, only solve small size problem instances.
	\item \cite{garey1979} \textcolor{RoyalBlue}{NP-Hard, and therefore (under the assumption that $P \neq NP$) there is no known algorithm that is able to find an optimal solution in polynomial time for every instance of the BPP.}
	\item \textcolor{RoyalBlue}{Instead, heuristics can be used to find a near optimal solution in a shorter amount of time}.
	\item discuss heuristics.
	\item Explain FF and FFD, online/offline \cite{eilon1971}, \cite{dosa2007} tight bound, optimal number of bins $k$ \cite{korf2002}, complexity $O(n \log n)$ \cite{coffman1984} (where n is the number of items to be packed).
	\item quickly produce a feasible solution, but may not be optimal.
	\item Use heuristics in this paper to find feasible solutions for the problem, why?
\end{itemize}

Notation and Formulating the problem
\begin{itemize}
	\item $n$ boxes, $i = 1, 2, ..., n$, same height, different widths.
	\item Each box $i$ has a total width $w(i)$.
	\item Each box $i$ has two scores, $i_1$ (the smaller score), and $i_2$ (the larger score), and widths $w(i_1), w(i_2)$, such that $w(i_1) \leq w(i_2)$.
	\item Score widths not necessarily equal, therefore box has two different positions.
	\item $\{i_1, i_2\}$ represents ``regular'' box $i$: $i_1$ on LHS of box.
	\item $\{i_2, i_1\}$ represents ``rotated'' box $i$: $i_2$ on LHS of box (box rotated by $180^{\circ}$).
	\item Boxes will be placed side by side.
	\item All box widths $w(i) > 0$ and all score widths $w(i_1), w(i_2) > 0$ $\forall$ $i = 1, 2, ..., n$.
	\item Finite number of strips $S_j$, $j = 1, 2, ...$.
	\item All strips $S_j$ have equal fixed length $l > 0$ $\forall$ $j = 1, 2, ...$.
	\item All strips $S_j$ have a residual length $r(j)$, initially $r(j) = l$ $\forall$ $j = 1,2, ...$, which reduces as boxes are added to each strip.
	\item Assumption: all box widths $w(i) \leq l, i = 1, 2, ..., n$
	\item Initially, $S_j = \emptyset$ $\forall$ $j = 1, 2, ...$.
	\item For all strips $S_j \neq \emptyset$, the score on the right hand side of the last box in the strip is called the ``free score'', denoted $f_j$, with width $w(f_j)$.
	\item The sum of two adjacent score widths must fulfil the minimum total score width constraint ... $t$ is the minimum total score width.
	\item The minimum total score width constraint must be fulfilled in order for the scoring knives to be able to score the boxes correctly.
	\item The \textit{minimum total score width} is denoted by $t, t > 0$ \textcolor{RoyalBlue}{usually 70mm in industry}. \textcolor{RoyalBlue}{The \textit{minimum total score width constraint} is $w(u_p) + w(v_q) \geq t$, $u \neq v, p, q = 1 or 2$.}
	\item Strips packed from left to right.
	\item $l > 0, w(i) > 0, w(i) \leq l$
	\item Need to pack the boxes onto the strips so mtswc is met and least number of strips used.
	\item Formal definition of Score-Constrained Bin Packing Problem
\end{itemize}

FFD Approx
\begin{itemize}
	\item Explain FFD modified for scored boxes (basic).
	\item Free score $f_j$, residual length $r(j)$.
	\item Figure: partially packed strip showing first box in regular position, free score, minimum total score width constraint met between boxes, residual length.
	\item Pseudocode
	\item Description of algorithm
	\item Complexity, discuss how algorithm only checks if current box meets constraint with free score on final box, boxes are not able to be rearranged, even though there may be another arrangement of the boxes that is feasible (lead on to next algorithm FFD using exact algorithm).	
\end{itemize}

FFD Using Exact Algorithm
\begin{itemize}
	\item Combine FFD with exact algorithm to find a feasible solution to the SCBPP
	\item Unlike FFD approx, where attempts are made to place each box at the end of a strip, this algorithm is able to rearrange the order of the boxes in a strip to find a feasible arrangement.
	\item Can find a solution for a set of boxes on a strip when it may not have possible to find one in ffdapprox
	\item Describe Becker algorithm, using graph theory, Hamiltonian path/cycle
	\item Notation for Exact algorithm:
	\item $n$ boxes including dummy box.
	\item $2n$ vertices including dominating vertices.
	\item Graph $G$ is an undirected graph with $2n$ vertices $v_i, i = 1, 2, ...,2n$.
	\item Vertices are weighted, $w(v_i)$.
	\item Vertices that represent either side of same box referred to as ``partners''
	\item $P$ contains the set of edges between partners, fixed, $|P| = n$.
	\item Edges in $P$ are weighted, corresponding to box width.
	\item $Q$ contains the set of edges between vertices that meet the minimum total score width constraint, excluding partners (these edges are not weighted).
	\item $G(V, P \cup Q)$ - each edge on $G$ is either in $P$ or $Q$, not both.
	\item $t$ still represents the minimum total score width.
	\item $M$ is the set of matching edges, $M \subset Q$, edges chosen using the matching algorithm (need $|M| = n$).
	\item number of cycles on $G$ after MTGMA =  ?
	\item $C_k$: sets of edges for connecting cycles.
	\item Section: Becker algorithm
	\item Combine alg with ffd
	\item Pseudocode
	\item Description of algorithm
	\item Complexity
\end{itemize}

Comparison/Results

Conclusion
\section{References}
\cite{becker2010}, \cite{becker2015}, \cite{coffman1978}, \cite{dosa2007}, \cite{eilon1971}, \cite{garey1979}, \cite{gilmore1961}, \cite{gilmore1963}, \cite{goulimis2004}, \cite{johnson1974}, \cite{karp1972}, \cite{korf2002}, \cite{lewis2009}, \cite{lewis2017}, \cite{lewis2011}, \cite{mahadev1994}, \cite{mahadev1995}, \cite{martello1990a}, \cite{martello1990b}

\cite{becker2010}: Twin Constrainted Hamiltonain Paths on Threshold Graphs 

\cite{becker2015}: A Heuristics for the MSSP

\cite{coffman1978}: An Application of Bin-Packing to Multiprocessor Scheduling

\cite{coffman1984}: Approximation Algorithms for Bin-Packing - An Updated Survey.
\begin{itemize}
	\item \textcolor{Plum}{``first fit and first fit decreasing algorithms have equivalent worst-case time complexities of O(n ln n) (cite coffman1984)''}
	\item \textcolor{Plum}{``we refer the interested reader to the excellent survery by (cite coffman1984), which includes a bibliography of more than one hundred titles''}
	\item \textcolor{Plum}{``A comprehensive review of various heuristic algorithms is provided in a recent survery by (cite coffman1984)''}
	\item \textcolor{Plum}{``(cite coffman1984) did an excellent survey on this problem, particularly on approximation algorithms and their asymptotic performance ratios''}
\end{itemize}

\cite{dosa2007}: The tight bound of first fit decreasing bin packing algorithm

\cite{eilon1971}: The loading problem. 
\begin{itemize}
	\item \textcolor{Plum}{``(cite eilon1971) presented a depth-first enumerative algorithm''}
	\item \textcolor{Plum}{``(cite eilon1971) discuss its(the BPP) application to the loading of vehicles (or other containers) with consignment''}
	\item \textcolor{Plum}{``the most well-known heuristics are the FFD and the BFD (cite eilon1971)''}
\end{itemize}

\cite{garey1979}: Computers and Intractibility - A Guide to the Theory of NP-Completeness. 
\begin{itemize}
	\item \textcolor{Plum}{``(cite garey1979) cite simple heuristics which can be shown to be noworse(but also no better) than a rather small multiplying factor above the optimal number of bins.''}
	\item \textcolor{Plum}{``the bpp problem is NP-Hard''}
\end{itemize}

\cite{gilmore1961}: A LP Approach to the CSP Part 1

\cite{gilmore1963}: A LP Approach to the CSP Part 2

\cite{goulimis2004}: Minimum Score Separation

\cite{johnson1974}: Worst case performance bounds for simple one-dimensional packing algorithms
\begin{itemize}
	\item \textcolor{Plum}{ ``(cite johnson1974) discuss three practical applications in computer science, which are table formatting, prepaging, and file allocation''}
\end{itemize}

\cite{karp1972}: Reducibility among combinatorial problems

\cite{korf2002}: A new algorithm for optimal bin packing. \textcolor{Plum}{``If the number of bins in a solution is equivalent to the expression (sum of all weights/ capacity) then that solution is an optimal solution to the problem instance (cite Korf2002)''} 

\cite{lewis2009}: A general-purpose hill-climbing method for order independent minimum grouping problems

\cite{lewis2017}: How to pack trapezoids

\cite{lewis2011}: An investigation into two bpp with ordering and orientation implications

\cite{mahadev1994}: Longest cycle in threshold graphs

\cite{mahadev1995}: Threshold graphs and related topics

\cite{martello1990a}: Knapsack problems.\textcolor{Plum}{ ``(cite martello1990) proposed a branch-and-bound based exact algorithm MTP for the BPP.''} 

\cite{martello1990b}: Lower bounds and reduction procedures for the bin packing problem

\section{Other}
\begin{itemize}
	\item complexity
\end{itemize}

\bibliographystyle{dcu}
\bibliography{includes/bibliography}






\end{document}